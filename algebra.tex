\documentclass{article}
\usepackage{soul} % Strike through
\usepackage{amsmath}
\usepackage{ulem}
\usepackage{amssymb}
\usepackage{polski}
\usepackage[utf8]{inputenc}
\usepackage{graphicx}
\usepackage{xcolor}
\usepackage{centernot}
\usepackage{amsthm}
\usepackage{csquotes, dirtytalk, marginnote, lipsum, scrextend, xcolor, graphicx, amssymb, amstext, amsmath, epstopdf, booktabs, verbatim, gensymb, geometry, appendix, natbib, lmodern}
\usepackage[pagestyles]{titlesec}
\usepackage{fancyhdr}
\usepackage{needspace}
\usepackage{etoolbox}
\usepackage{mparhack}
\usepackage{psvectorian}
\geometry{b5paper, bottom=2.5cm}

\usepackage[utf8]{inputenc}
\usepackage[polish]{babel}
\usepackage{fontspec}
\usepackage{tocloft}
\usepackage[perpage]{footmisc}
\usepackage{enumitem}
\usepackage{sectsty}

\theoremstyle{definition}
\newtheorem{definition}{Definicja}[section]
\newtheorem{example}{Przykład}[section]
\newtheorem{exercise}{Zadanie}[section]

\date{MM XX V.}
\author{Szymon Zyguła.}
\title{Podstawy Algebry Abstrakcyjnej.}


\begin{document}

\maketitle

\begin{enumerate}
	\item Kongruencje
	\item Jądra
	\item Tw. o homomorfizmach.
\end{enumerate}

\section{Struktury algebraiczne.}

Zajmiemy się teraz zdefiniowaniem tak zwanych \textbf{struktur algebraicznych}.
Będą to zbiory z działaniami, które spełniają pewne własności.
Znanym nam już przykładem{\mbox{---}}do którego jednak przejdziemy znacznie później---są przestrzenie liniowe.

W tekście zakładamy, że $0 \in \mathbb{N}$.

\subsection{Podstawowe struktury.}

\begin{definition}
	\textbf{Magmą} nazywamy parę $(A, \star)$, gdzie $A$ jest zbiorem, a $\star: A^2 \to A$ jest dwuargumentową funkcją na A (nazywaną też działaniem). Działanie $\star$ na $x, y \in A$ zapisujemy $x \star y$.
\end{definition}
Często magmę $(A, \star)$ będziemy utożsamiać ze zbiorem $A$ i mówić, że $A$ jest magmą.
Podobny zabieg wykonamy dla wszystkich kolejnych definiowanych przez nas struktur.

\begin{example}
	$(\mathbb{N}_+, +)$ (przez $\mathbb{N}_+$ rozumiemy dodatnie liczby naturalne) jest magmą.
\end{example}

\begin{example}
	$(\mathbb{Z}, +)$ jest magmą.
\end{example}

\begin{example}
	$(\mathbb{Q}_+, \div)$ (przez $\mathbb{Q}_+$ rozumiemy dodatnie liczby wymierne) jest magmą.
\end{example}

\begin{definition}
	\textbf{Półgrupą} nazywamy magmę $(A, \star)$, że działanie $\star$ jest łączne. To znaczy, że
	\begin{equation*}
		\forall_{x,y,z \in A}\; a \star (b \star c) = (a \star b) \star c.
	\end{equation*}
\end{definition}

\begin{example}
	$(\mathbb{N}_+, +)$ jest półgrupą.
\end{example}

\begin{example}
	$(\mathbb{Z}, +)$ jest półgrupą.
\end{example}

\begin{example}
	$(\mathbb{Q}_+, \div)$ \textit{nie} jest półgrupą, ponieważ przykładowo
	\begin{equation*}
		1 \div (2 \div 2) = 1 \neq \frac{1}{4} = (1\div2)\div2.
	\end{equation*}
\end{example}

Dzięki łączności,
	działania w półgrupach nie wymagają nawiasów.
Przykładowo, jeśli $(A, \cdot)$ jest półgrupą i $a, b, c, d \in A$,
	to zapis
\begin{equation}
	a \cdot b \cdot c \cdot d
\end{equation}
	jest jednoznaczny, bo
\begin{equation}
		a \cdot (b \cdot (c \cdot d)) = 
		(a \cdot b) \cdot (c \cdot d) =
		(a \cdot ((b \cdot c) \cdot d) = \dots
\end{equation}
	co niekoniecznie zachodzi w magmie.
Własność tę można udowodnić indukcyjnie.

Jeśli w półgrupie działanie oznaczamy przez $\cdot$,
	to zwykle nie będziemy go zapisywać.
To znaczy, zamiast pisać $x \cdot y$ będziemy pisać $xy$.

\begin{definition}
	\textbf{Monoidem} nazywamy trójkę $(A, \star, e)$,
		że $(A, \star)$ jest półgrupą,
		a $e\in A$ jest \textbf{elementem neutralnym} działania $\star$,
		t.j. takim, że
	\begin{equation}
		\forall_{x \in A}\; e \star x = x = x \star e.
	\end{equation}
\end{definition}

Element $e$ nazywamy zwykle \textbf{jednością},
	\textbf{jedynką} lub \textbf{zerem},
	zależnie od kontekstu.
Jeśli element ten będzie wiadomy z kontekstu,
	to nie będziemy o nim wspominać

\begin{example}
	$(\mathbb{Z}, +, 0)$ jest monoidem.
\end{example}

\begin{example}
	$(\mathbb{N}_+, +)$ \textit{nie} jest monoidem, bo nie jest trójką, tylko parą.
	Ale nawet gdybyśmy chcieli,
		to nie zrobimy z tego zbioru i działania monoidu,
		bo nie ma dodatniej liczby naturalnej $n$,
		że $n + a = a$ dla każdego $a \in \mathbb{N}_+$.
\end{example}

\begin{example}
	Ciągi znaków (\textit{stringi}) z operacją konkatenacji i pustym ciągiem tworzą monoid.
\end{example}

\begin{example}
	$\left(\mathbb{R}^{2 \times 2}, \cdot, \left[\begin{matrix} 1 & 0 \\ 0 & 1 \end{matrix}\right]\right)$ jest monoidem.
\end{example}

\noindent\sout{
	\textbf{Przykład 1.1.?} Monada jest monoidem w kategorii endofunktorów.
}

Jeśli ktoś kiedyś słyszał lub usłyszy o powyższym stwierdzeniu,
	często żartobliwie przytaczanym jako ,,proste'' wytłumaczenie,
	czym jest monada, to zapewniam, że jest ono prawdziwe.
Chodzi tam jednak o inny monoid,
	znacznie uogólniony w stosunku do naszego.

\begin{definition}
    \textbf{Grupą} nazywamy czwórkę $(A, \star, e, \mathrm{inv})$,
		że $(A, \star, e)$ jest monoidem,
		a $\mathrm{inv}: A \to A$ jest funkcją przypisującą każdemu elementowi jego \textbf{odwrotność},
		to znaczy spełniającą warunek
	\begin{equation*}
		\forall_{x \in A}\; {x\star \mathrm{inv}(x) = e = \mathrm{inv}(x) \star x}.
	\end{equation*}
\end{definition}
Podobnie jak w przypadku monoidu,
	jeśli jedność i funkcja odwrotności będzie oczywista,
	to nie będziemy o niej wspominać.

\begin{example}
	$(\mathbb{Q}, +, 1, (\cdot^{-1}))$ jest grupą.
\end{example}

\begin{example}
	Używając uproszczonej terminologii, zbiór
	\begin{equation}
		\{A \in \mathbb{R}^{2 \times 2} | \det(A) \neq 0 \}
	\end{equation}
		ze standardowym mnożeniem macierzy stanowi grupę.
\end{example}

\begin{example}
	Zbiór $\mathbb{R}^{n \times n}$ dla ustalonego $n > 1$ nie stanowi grupy
		ze standardowym mnożeniem macierzy,
		gdyż nie każdy element ma odwrotność.
	Ten sam zbiór z dodawaniem macierzy jest już jednak grupą---odwrotnościami są negacje macierzy.
\end{example}

\subsubsection{Przemienność.}
Jeśli w którejś ze zdefiniowanych przez nas struktur działanie $\star$ jest przemienne, to znaczy
\begin{equation}
	\forall_{x,y \in A}\; x \star y = y \star x,
\end{equation}
to strukturę tę nazywamy przemienną.
Mamy więc \textbf{przemienne magmy}, \textbf{przemienne półgrupy}, \textbf{przemienne monoidy} i \textbf{przemienne grupy}.
Grupy przemienne nazywa się jednak zwykle \textbf{grupami abelowymi},
od nazwiska norweskiego matematyka Nielsa Henrika Abela, 1802--1829.
,,Abelowy'' piszemy jednak z małej litery (z wyjątkiem początku zdania), gdyż jest to przymiotnik.

\begin{example}
	$\mathbb{N}$ z dodawaniem jest półgrupą przemienną.
\end{example}

\begin{example}
	$\mathbb{Q}$ z mnożeniem jest grupą abelową.
\end{example}

\begin{example}
	Zbiór ciągów znakowych z konkatenacją \textit{nie} jest monoidem przemiennym---w konkatenacji kolejność ma znaczenie.
\end{example}

\subsection{Złożone struktury.}
Używając zdefiniowanych przez nas prostych struktur algebraicznych możemy definiować bardziej złożone,
	składające się z wielu działań (ale nadal tylko jednego zbioru).

\begin{definition}
	\textbf{Pierścieniem} nazywamy szóstkę $(P, +, 0, \mathrm{neg}, \cdot, 1)$,
		że $(P, +, 0, \mathrm{neg})$ jest grupą abelową,
		a $(P, \cdot, 1)$ jest monoidem,
		oraz zachodzą równania
		\begin{align}
			\label{dist1}\forall_{a, b, c \in P} a \cdot (b + c) &= a \cdot b + a \cdot c \\
			\label{dist2}\forall_{a, b, c \in P} (a + b) \cdot c &= a \cdot c + b \cdot c
		\end{align}
\end{definition}

W pierścieniu mamy już nie jedno,
	a dwa działania.
Tworzą one niezależnie od siebie grupę abelową i monoid,
	ale dodatkowo łączy jest \textbf{rozdzielność},
	czyli równania \eqref{dist1} i \eqref{dist2}.

\begin{example}
	Liczby całkowite z dodawaniem i mnożeniem tworzą pierścień.
	Jest to podstawowy przykład i inspiracja dla definicji pierścienia.
\end{example}

\begin{example}
	Niech $P_n(\mathbb{R})$ będzie zbiorem wielomianów o współczynnikach z $\mathbb{R}$ o
		stopniu conajwyżej $n$.
	Wtedy $P_n(\mathbb{R})$ z dodawaniem i mnożeniem wielomianów tworzy pierścień.
\end{example}

Dodatkowo możemy zdefiniować,
	że pierścień $(P, +, 0, \mathrm{neg}, \cdot, 1)$ jest przemienny,
	jeśli działanie $\cdot$ jest przemienne.
Przemienność $+$ jest natomiast wymagana już przez samą definicję pierścienia.

\begin{definition}
	\textbf{Ciałem} nazywamy siódemkę $(K, +, 0, \mathrm{neg}, \cdot, 1, \mathrm{inv})$,
		że $(K, +, 0, \mathrm{neg}, \cdot, 1)$ jest pierścieniem,
		a $(K \\ \{0\}, \cdot, 1, \mathrm{inv})$ jest grupą abelową.
\end{definition}

\begin{example}
	$\mathbb{Q}$ z dodawaniem i dzieleniem tworzy ciało.
\end{example}

\begin{example}
	$\mathbb{R}$ z dodawaniem i dzieleniem tworzy ciało.
\end{example}

\begin{example}
	$\mathbb{Z}$ z dodawaniem i dzieleniem \textit{nie} tworzy ciała---w liczbach całkowitych
		odwrotność względem mnożenia ma tylko $1$.
\end{example}

\subsection{Zadania.}

\begin{exercise}
	Pokaż, że zbiór macierzy $\mathbb{R}^{2 \times 2}$ z mnożeniem
		jest magmą, półgrupą, monoidem.
	Pokaż, że nie jest grupą.
\end{exercise}

\begin{exercise}
	Podaj przykład grupy, która nie jest przemienna.
\end{exercise}

\begin{exercise}
	Sprawdź, czy $(\mathbb{C}, \cdot)$ jest grupą.
\end{exercise}

\begin{exercise}
	Czy $\mathbb{C}$ ze standardowym mnożeniem jest monoidem? Jeśli tak, to jaki jest element neutralny? Czy jest to grupa?
\end{exercise}

\begin{exercise}
	Niech $A$ będzie dowolnym zbiorem.
	Które z definicji struktur algebraicznych spełnia zbiór funkcji $A \to A$ z operacją składania funkcji?
\end{exercise}

\begin{exercise}
	Sprawdź,
		które z definicji struktur algebraicznych spełnia zbiór ścieżek na płaszczyźnie
		(t.j. funkcji $[0, 1] \to \mathbb{R}^2$) z operacją łączenia ich swoimi końcami
		(tzn. dwie ścieżki możemy połączyć w jedną,
			wtedy i tylko wtedy gdy pierwsza kończy się tam,
			gdzie druga się zaczyna).
\end{exercise}

\begin{exercise}
	Udowodnij,
		że $\mathbb{R}^{2 \times 2}$ ze standardowym dodawaniem i mnożeniem macierzy tworzy pierścień.
\end{exercise}

\begin{exercise}
	Zapisz definicję ciała, nie używając definicji pierścienia.
\end{exercise}

\begin{exercise}
	Udowodnij, że liczby zespolone $\mathbb{C}$ są ciałem.
\end{exercise}

\begin{exercise}
	Dlaczego zdefiniowaliśmy przemienne warianty każdej ze struktur, ale nie ciał?
\end{exercise}

\begin{exercise}
    Niech $(A, \cdot, e, \mathrm{inv})$ będzie dowolną grupą.
    Udowodnij, że dla każdego $x \in A$ zachodzi ${x^{-1}}^{-1} = x$.
\end{exercise}

\begin{exercise}
    Niech $(A, \cdot, e)$ będzie dowolnym monoidem.
	Niech $y \in A$.
    Udowodnij, że jeśli dla każdego $x \in A$ zachodzi
	\begin{equation}
        yx = x,
	\end{equation}
	lub
	\begin{equation}
        xy = x,
	\end{equation}
    to $y = e$.
    Pokazuje to,
		że mając dane działanie,
		istnieje conajwyżej jeden element neutralny.
\end{exercise}

\begin{exercise}
    Niech $(A, \cdot, e, \mathrm{inv})$ będzie dowolną grupą.
	Niech $x \in A$.
    Udowodnij, że jeśli dla pewnego $y \in A$ zachodzi
	\begin{equation}
        yx = e,
	\end{equation}
	lub
	\begin{equation}
        xy = e,
	\end{equation}
    to $y = \mathrm{inv}(x)$.
    Pokazuje to,
		że mając dane działanie i element neutralny,
        dla każdego elementu istnieje conajwyżej jeden sposób zdefiniowania odwrotności.
\end{exercise}

\section{Algebra Uniwersalna.}
Wejdziemy teraz na nieco wyższy poziom abstrakcji.
Szybko jednak zobaczymy,
że znacząco ułatwi nam to posługiwanie się strukturami,
które zdefiniowaliśmy.

Na czym polega problem?
Poznane przez nas struktury algebraiczne wyglądają wszystkie bardzo podobnie.
Mamy zbiór, pewne stałe i operacje wykonywane na tym zbiorze.
Możemy się już domyślać,
	że będziemy mieli wiele bardzo podobnych twierdzeń i definicji dla każdej ze zdefiniowanych struktur.

Przykładowo,
	w algebrze występuje pojęcie \textbf{homomorfizmu}---funkcji $h: A \to B$ między dwoma strukturami
	tego samego typu (np. między dwoma grupami, dwoma monoidami itd.),
	które dodatkowo spełniają pewne warunki.
Zwykle jednym z nich jest (jeśli $(A, \star, \dots)$ i $(B, \bullet, \dots)$ są jakimiś strukturami)
\begin{equation}
	h(a\star b) = h(a) \bullet h(b).
\end{equation}
Ale definicje homomorfizmów mają też więcej wymagań.
W przypadku monoidów $(A, \star, e_A)$, $(B, \star, e_B)$ chcemy aby zachodziło też
\begin{equation}
	h(e_A) = e_B.
\end{equation}
W przypadku grup podobna własność musi zachodzić też dla odwracalności.

Najczęściej homomorfizmy definiuje się osobno dla każdej struktury.
W trakcie studiów spotkaliśmy się już z przykładem homomorfizmu---mowa o przekształceniach liniowych
	(przestrzenie liniowe też są strukturą algebraiczną!).
Nasuwa się jednak pytanie: czy nie dałoby się uogólnić tego,
	abyśmy mieli tylko jedną definicję?
Chcielibyśmy mieć coś takiej postaci:
\\\\
\noindent \textbf{Definicja}.
	\textbf{Homomorfizmem} struktur $(A, \dots)$ i $(B, \dots)$ nazywamy funkcję $h: A \to B$,
	spełniającą warunki \dots
\\\\
Ciężko jednak powiedzieć, co należałoby wstawić w miejsce kropek, aby sformalizować tę definicję.
Podobna sytuacja zajdzie,
    kiedy zechcemy zacząć definiować
	\textbf{podprzestrzenie liniowe}, \textbf{podgrupy}, \textbf{podciała}, \textbf{podmonoidy} itd.
Definicje tych obiektów byłyby do siebie bardzo podobne.
Jak jednak zawrzeć je wszystkie w jednej definicji?
Odpowiedzi na te pytania dostarcza nam dziedzina algebry zwana \textit{Algebrą Uniwersalną}.

\subsection{Podstawowe definicje.}

\begin{definition}
	Niech $A$ będzie zbiorem.
	Mówimy, że funkcja $f: A^n \to A$ ma \textbf{arność} $n$ (od końcówki \textit{-arny},
	a to z łacińskiej końcówki \textit{-aris},
	w analogii do bin\textit{arny}, tern\textit{arny}, itd.).
\end{definition}

Ciekawa i przydatna własność zachodzi, gdy w powyższej definicji $n = 0$.
Zbiór $A^0$ definiuje się jako zbiór jednoelementowy (co jest tym jednym elementem nie ma znaczenia).
Funkcja $f:A^0 \to A$ jest więc funkcją ze zbioru jednoelementowego w zbiór $A$.
Możemy więc na nią patrzeć jak na pojedynczą stałą (bo zbiór wartości też jest jednoelementowy).

\begin{definition}
	\textbf{Algebrą} nazywamy trójkę $(\Sigma, \alpha, E)$,
	gdzie
	\begin{itemize}
		\item $\Sigma$ jest zbiorem, zwanym zbiorem \textbf{symboli},
		\item $\alpha: \Sigma \to \mathbb{N}$ jest funkcją, która każdemu symbolowi przypisuje arność,
		\item $E$ jest zbiorem równań używających zmienny i symboli z $\Sigma$ jako funkcji.
	\end{itemize}
	Parę $(\Sigma, \alpha)$ nazywamy \textbf{sygnaturą} algebry $(\Sigma, \alpha, E)$.
\end{definition}

Definicja ta nie jest w pełni formalna (czym są ,,symbole"? czym są ,,zmienne"?
	co znaczy, że elementów $\Sigma$ używamy jako funkcji?
	jak możemy czemuś, co nie jest funkcją, przypisać arność?).
Sformalizowanie jej jednak wprowadziłoby więcej zamieszania niż pożytku,
	gdyż wymagałoby wprowadzenia mało istotnych, technicznych szczegółów.
Zamiast tego przejdziemy rozjaśniającego przykładu.

\begin{example}
	Trójka $(\Sigma, \alpha, E)$, gdzie
	\begin{itemize}
		\item $\Sigma = \{f, g, e\}$,
		\item $\alpha: f \mapsto 2, g \mapsto 2, e \mapsto 0$,
		\item $E = \{g(x, y) = g(y, x), f(e, x) = x, f(x, e) = x\}$
	\end{itemize}
	jest algebrą.
\end{example}
Jak możemy interpretować, czym jest taka algebra?
Jest to nic innego jak definicja pewnej struktury algebraicznej,
gdzie mamy dwie operacje dwuargumentowe $f, g$ i jedną wyróżnioną stałą $e$,
że $g$ jest przemienne, a $e$ jest elementem neutralnym dla $f$.

Tłumacząc powyższy zapis na tradycyjną definicję, moglibyśmy napisać następująco.
\begin{definition}
	\textbf{Gałganem} nazywamy czwórkę $(A, f, g, e)$,
		gdzie $A$ jest zbiorem,
		$f, g: A^2 \to A$ są funkcjami dwuargumentowymi,
		$e \in A$ (lub, co w pewnym sensie jest równoważne, $e: \{0\} \to A$)
		i spełnione są następujące warunki:
	\begin{itemize}
		\item $\forall_{x\in A}\; f(e, x) = x,$
		\item $\forall_{x\in A}\; f(x, e) = x,$
		\item $\forall_{x,y\in A}\; g(x, y) = g(y, x).$
	\end{itemize}
\end{definition}

Zwykle zapis będziemy jednak upraszczać,
	podając jednocześnie zbiór symboli $\Sigma$ i funkcję $\alpha$ za jednym razem,
	n.p. w postaci
\begin{equation}
	\Sigma = \{f^2, g^2, e\},
\end{equation}
t.j. podając w definicji symboli ich arność w indeksie górnym, i pomijając ją dla $0$.
Idąc tą konwencją będziemy też zamiast $(\Sigma, \alpha, E)$ pisać $(\Sigma, E)$.

Algebry odpowiadają więc definicjom struktur.
Co odpowiada jednak obiektom, spełniającym te definicje?

\begin{definition}
    \textbf{Modelem algebry} $(\Sigma, \alpha, E)$ nazywamy zbiór $A$ (zwany \textbf{nośnikiem}),
	wraz z interpretacją symboli z $\Sigma$ jako funkcje o arnościach definiowanych przez $\alpha$ i spełniających równania $E$.
\end{definition}
Znowu nie jest to całkowicie formalna definicja,
na nasze potrzeby jednak w pełni wystarczy.

\begin{example}
	Zbiór $\mathbb{R}$ z działaniami $f(x, y) = x + y$, $g(x, y) = xy$ jest modelem gaugana.
\end{example}

Możemy teraz definiować nasze struktury w taki sam sposób jak gaugana.
\begin{definition}
	\textbf{Monoidem} nazywamy dowolny model algebry
	\begin{equation*}
		(\{\star^2, e\}, \{ x \star e = x, e \star x = x \}).
	\end{equation*}
\end{definition}

Ważne, aby rozróżniać teraz dwa pojęcia: algebrą jest \textit{definicja} monoidu.
Modelem algebry jest natomiast dowolny monoid,
	czyli obiekt,
	który spełnia tę definicję.

Algebra uniwersalna nie jest jednak taka uniwersalna,
	jak byśmy chcieli.
Jedna z podanych przez nas w poprzedniej sekcji definicji nie stanowi algebry.
Mowa tutaj o ciałach.
Problemem jest to,
	że funkcja $\mathrm{inv}$ nie musi być określona dla elementu neutralnego dodawania w ciele.
Jest to warunek,
	którego nie da się sformułować używając pojęcia algebry.
Z tego powodu ciała będziemy musieli traktować osobno od pozostałych struktur.

\subsection{Zadania.}

\begin{exercise}
	Zapisz definicję magmy w postaci algebry.
\end{exercise}
\begin{exercise}
	Zapisz definicję półgrupy w postaci algebry.
\end{exercise}
\begin{exercise}
	Zapisz definicję grupy w postaci algebry.
\end{exercise}
\begin{exercise}
	Zapisz definicję grupy abelowej w postaci algebry.
\end{exercise}
\begin{exercise}
	Zapisz definicję pierścienia w postaci algebry.
\end{exercise}

\section{Pewne definicje związane z algebrami.}

\begin{definition}
    \textbf{Podmodelem modelu $M$ algebry $(\Sigma, E)$} nazywamy podzbiór $N \subseteq M$,
		który jest modelem algebry $(\Sigma, E)$ z tymi samymi operacjami co $M$.
\end{definition}

\begin{example}
	$2\mathbb{N}$, czyli liczby parzyste naturalne stanowią podpółgrupę liczb naturalnych z dodawaniem.
\end{example}
\begin{example}
	$\mathbb{Q}$ stanowi podgrupę przemienną grupy $\mathbb{R}$ z mnożeniem.
\end{example}

\begin{definition}
    \textbf{Homomorfizmem} (lub czasami \textbf{morfizmem}) nazywamy
\end{definition}

Z pewnych niewyjaśnionych powodów,
    w przypadku homomorfizmów nie używa się pojęć takich jak iniekcja, bijekcja i surjekcja.
Zamiast tych pochodzących z francuskiego wyrazów mamy greckie odpowiedniki.

\begin{definition}
    Homomorfizm nazywamy
    \begin{itemize}
        \item \textbf{epimorfizmem}, jeśli jest surjekcją,
        \item \textbf{monomorfizmem}, jeśli jest iniekcją,
        \item \textbf{izomorfizmem}, jeśli jest bijekcją,
        \item \textbf{endomorfizmem}, jeśli jego dziedzina i kodziedzina są równe, t.j. jest postaci $h:A \to A$,
        \item \textbf{automorfizmem}, jeśli jest endomorfizmem i izomorfizmem.
    \end{itemize}
\end{definition}

\begin{definition}
	Mówimy, że modele $A, B$ algebry $(\Sigma, E)$ są izomorficzne,
		jeśli istnieje izomorfizm $f: A \to B$.
	Fakt ten zapisujemy $A \cong B$.
\end{definition}
Jeśli dwa modele są izomorficzne, to w pewnym sensie są identyczne---różnią się tylko formą zapisu.
Poza tym wszystkie algebraiczne własności,
	które zachodzą w jednym modelu,
	zachodzą też w drugim.
W niektórych dziedzinach matematyki izomorficzne modele identyfikuje się tak bardzo,
	że zamiast używać znaku $A \cong B$ używa się zwykłej równości $=$.
My jednak zostaniemy przy $\cong$.

\begin{example}
	$\mathbb{R}$ z dodawaniem jest grupą abelową izomorficzną z grupą $\mathbb{R}^{1 \times 1}$ z dodawaniem macierzy.
\end{example}

\begin{example}
	$\mathbb{Z}$ z dodawaniem \textit{nie} jest izomorficzne z $\mathbb{Z}$ z mnożeniem.
\end{example}

\begin{definition}
    Wolnym modelem algebry $(\Sigma, E)$ generowanym przez zbiór $G$ (tak zwany \textbf{zbiór generatorów}),
    nazywamy model algebry $(\Sigma, E)$, którego nośnikiem jest zbiór wszystkich wyrażeń,
		które utworzyć możemy z elementów $\Sigma$ i $G$, traktują elementy $\Sigma$ jak
		funkcje z odpowiednimi arnościami.
	Dodatkowo,
		dwa wyrażenia uznajemy za równe,
		jeśli jedno możemy przekształcić w drugie używając równań z $E$.
\end{definition}
Mówimy też, że model jest \textit{nad zbiorem $G$}.
Znowu nie jest to w pełni formalna definicja,
	jednak bardzo intuicyjnie oddająca istotę wolnych modeli.
Definicja w pełni formalna wymagałaby albo wprowadzenia pewnych formalnych pojęć,
	których nie będziemy wprowadzać (ilorazy modeli przez podmodele),
	albo byłaby bardzo abstrakcyjna.

Zwykle jako zbiór $G$ bierze się pewne formalne symbole, czyli jakieś litery.

\begin{example}
	Grupą wolną na zbiorze $\{a\}$ nazywamy grupę, której elementami są
	$e, a, a^{-1}, a^2, a^{-2}, \dots$.
	Zachodzą w niej m.in. równości
	\begin{align}
		aa{^-1} &= e \\
		eae &= a \\
		a^{-1}a &= e
	\end{align}
\end{example}

W pewnym sensie grupą wolną na zbiorze $G$ jest najprostrza grupa,
	która zawiera całe $G$ jako swoje elementy.
Najprostsza,
	bo nie zachodzą w niej inne równania,
	niż te,
	które stanowią aksjomaty grupy,
	oraz te,
	które można z nich wyprowadzić.

\begin{example}
	$\mathbb{Z}$ wraz z dodawaniem stanowi wolną grupę przemienną nad zbiorem $\{1\}$
\end{example}

\begin{example}
	$\mathbb{Z}$ wraz z mnożeniem nie stanowi wolnej grupy przemiennej,
		bo dla dowolnego elementu $x$ mamy
		\begin{equation}
		0x = 0
		\end{equation}
		czego nie możemy wyprowadzić z aksjomatów grupy przemiennej.
\end{example}

Podobnie sprawa ma się z innymi wolnymi modelami algebr.
W wolnym monoidzie spełnione jest tylko
	

\begin{definition}
    Prezentacją modelu
\end{definition}

\subsection{Zadania.}

\begin{exercise}
	Zapisz definicję homomorfizmu monoidów nie używając pojęć algebry uniwersalnej.
\end{exercise}

\begin{exercise}
	Zapisz definicję homomorfizmu grup nie używając pojęć algebry uniwersalnej.
\end{exercise}

\begin{exercise}
	Sprawdź, czy następujące funkcje
	\begin{enumerate}
		\item $f(x) = 2x$,
		\item $g(x) = 2 + x$
		\item $h(x) = x^2$,
	\end{enumerate}
	są endomorfizmami w grupie
	\begin{enumerate}
		\item $\mathbb{Z}$ z mnożeniem,
		\item $\mathbb{Z}$ z dodawaniem.
	\end{enumerate}
\end{exercise}

\begin{exercise}
	Niech $(A, +), (B, \cdot)$ będą dowolnymi grupmi.
	Udowodnij,
		że jeśli $f: A \to B$ jest funkcją spełniającą
	\begin{equation}
		f(x + y) = f(x) + f(y)
	\end{equation}
	dla każdych $x, y \in A$, to $f$ jest homomorfizmem grup.
\end{exercise}

\begin{exercise}
	Niech $A$ będzie grupą abelową, a $B$ dowolną grupą.
	Udowodnij, że jeśli $f: A \to B$ jest epimorfizmem,
		to $B$ jest grupą abelową.
\end{exercise}

\begin{exercise}
	Udowodnij, że jeśli $A, B$ są zbiorami i $|A| = |B|$,
		to grupa wolna nad $A$ i grupa wolna nad $B$ są izomorficzne.
\end{exercise}

\begin{exercise}
	Udowodnij, że jeśli $A, B$ są zbiorami i $|A| = |B|$,
		a $\mathfrak{A} = (\Sigma, E)$ jest dowolną algebrą,
		to dowolne dwa wolne modele $\mathfrak{A}$ nad odpowiednio $A$ i $B$ są izomorficzne.
\end{exercise}

\section{Kongruencje.}
Przejdziemy teraz do nieco bardziej abstrakcyjnego pojęcia,
	tak zwanych kongruencji.
Mówiąc nieformalnie,
	są to relacje równoważności na modelach algebr,
	które są spójne ze strukturą tych modeli.



\section{Twierdzenia o homomorfizmach.}

\section{Grupy.}

Pochylimy się teraz nieco bardziej szczegółowo nad grupami,
gdyż stanowią one jedną z najważniejszych i najgłębiej badanych struktur algebraicznych.



\section{Przestrzenie liniowe.}

\end{document}
