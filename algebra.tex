\documentclass{article}
\usepackage{soul} % Strike through
\usepackage{amsmath}
\usepackage{ulem}
\usepackage{amssymb}
\usepackage{polski}
\usepackage[utf8]{inputenc}
\usepackage{graphicx}
\usepackage{xcolor}
\usepackage{centernot}
\usepackage{amsthm}
\usepackage{csquotes, dirtytalk, marginnote, lipsum, scrextend, xcolor, graphicx, amssymb, amstext, amsmath, epstopdf, booktabs, verbatim, gensymb, geometry, appendix, natbib, lmodern}
\usepackage[pagestyles]{titlesec}
\usepackage{fancyhdr}
\usepackage{needspace}
\usepackage{etoolbox}
\usepackage{mparhack}
\usepackage{psvectorian}
\geometry{b5paper, bottom=2.5cm}

\usepackage[utf8]{inputenc}
\usepackage[polish]{babel}
\usepackage[oldstyle]{kpfonts}
\usepackage{fontspec}
\usepackage{tocloft}
\usepackage[perpage]{footmisc}
\usepackage{enumitem}
\usepackage{sectsty}

\theoremstyle{definition}
\newtheorem{definition}{Definicja}[section]
\newtheorem{example}{Przykład}[section]
\newtheorem{exercise}{Zadanie}[section]

\date{MM XX V.}
\author{Szymon Zyguła.}
\title{Podstawy Algebry Abstrakcyjnej.}


\begin{document}

\maketitle

\section{Struktury algebraiczne.}

Zajmiemy się teraz zdefiniowaniem tak zwanych \textbf{struktur algebraicznych}.
Będą to zbiory z działaniami, które spełniają pewne własności.
Znanym nam już przykładem---do którego jednak przejdziemy znacznie później---są przestrzenie liniowe.

W tekście zakładamy, że $0 \in \mathbb{N}$.

\subsection{Podstawowe struktury.}

\begin{definition}
	\textbf{Magmą} nazywamy parę $(A, \star)$, gdzie $A$ jest zbiorem, a $\star: A^2 \to A$ jest dwuargumentową funkcją na A (nazywaną też działaniem). Działanie $\star$ na $x, y \in A$ zapisujemy $x \star y$.
\end{definition}
Często magmę $(A, \star)$ będziemy utożsamiać ze zbiorem $A$ i mówić, że $A$ jest magmą.
Podobny zabieg wykonamy dla wszystkich kolejnych definiowanych przez nas struktur.

\begin{example}
	$(\mathbb{N}_+, +)$ (przez $\mathbb{N}_+$ rozumiemy dodatnie liczby naturalne) jest magmą.
\end{example}

\begin{example}
	$(\mathbb{Z}, +)$ jest magmą.
\end{example}

\begin{example}
	$(\mathbb{Q}_+, \div)$ (przez $\mathbb{Q}_+$ rozumiemy dodatnie liczby wymierne) jest magmą.
\end{example}

\begin{definition}
	\textbf{Półgrupą} nazywamy magmę $(A, \star)$, że działanie $\star$ jest łączne. To znaczy, że
	\begin{equation*}
		\forall_{x,y,z \in A}\; a \star (b \star c) = (a \star b) \star c.
	\end{equation*}
\end{definition}

\begin{example}
	$(\mathbb{N}_+, +)$ jest półgrupą.
\end{example}

\begin{example}
	$(\mathbb{Z}, +)$ jest półgrupą.
\end{example}

\begin{example}
	$(\mathbb{Q}_+, \div)$ \textit{nie} jest półgrupą, ponieważ przykładowo
	\begin{equation*}
		1 \div (2 \div 2) = 1 \neq \frac{1}{4} = (1\div2)\div2.
	\end{equation*}
\end{example}


\begin{definition}
	\textbf{Monoidem} nazywamy trójkę $(A, \star, e)$, że $(A, \star)$ jest półgrupą, a $e\in A$ jest \textbf{elementem neutralnym działania $\star$}, t.j. takim, że
	\begin{math}
		\forall_{x \in A}\; e \star x = \star x = x \star e.
	\end{math}
\end{definition}

\begin{example}
	$(\mathbb{Z}, +, 0)$ jest monoidem.
\end{example}

\begin{example}
	$(\mathbb{N}_+, +)$ \textit{nie} jest monoidem, bo nie jest trójką, tylko parą.
	Ale nawet gdybyśmy chcieli, to nie zrobimy z tego zbioru i działania monoidu, bo nie ma dodatniej liczby naturalnej $n$, że $n + a = a$ dla każdego $a \in \mathbb{N}_+$.
\end{example}

\begin{example}
	Ciągi znaków (\textit{stringi}) z operacją konkatenacji i pustym ciągiem tworzą monoid.
\end{example}

\begin{example}
	$(\mathbb{R}^{2 \times 2}, \cdot, \left[\begin{matrix} 1 & 0 \\ 0 & 1 \end{matrix}\right])$ jest monoidem.
\end{example}


\noindent\sout{
	\textbf{Przykład 1.1.?} Monada jest monoidem w kategorii endofunktorów.
}

Jeśli ktoś kiedyś słyszał lub usłyszy o powyższym stwierdzeniu, często żartobliwie przytaczanym jako ,,proste" wytłumaczenie, czym jest monada, to zapewniam, że jest ono prawdziwe. Chodzi tam jednak o inny monoid, znacznie uogólniony w stosunku do naszego.

\begin{definition}
    \textbf{Grupą} nazywamy czwórkę $(A, \star, e, (\cdot^{-1}))$, że $(A, \star, e)$ jest monoidem, a $(\cdot^{-1}): A \to A$ jest funkcją przypisującą każdemu elementowi jego \textbf{odwrotność}, to znaczy spełniającą warunek
	\begin{equation*}
		\forall_{x \in A}\; {x\star x^{-1} = e = x^{-1} \star x}.
	\end{equation*}
\end{definition}

\subsubsection{Przemienność.}
Jeśli w którejś ze zdefiniowanych przez nas struktur działanie $\star$ jest przemienne, to znaczy
\begin{equation}
	\forall{x,y \in A}\; x \star y = y \star x,
\end{equation}
to strukturę tę nazywamy przemienną.
Mamy więc \textbf{przemienne magmy}, \textbf{przemienne półgrupy}, \textbf{przemienne monoidy} i \textbf{przemienne grupy}.
Grupy przemienne nazywa się jednak zwykle \textbf{grupami abelowymi},
od nazwiska norweskiego matematyka Nielsa Henrika Abela, 1802--1829.
,,Abelowy" piszemy jednak z małej litery (z wyjątkiem początku zdania), gdyż jest to przymiotnik.

\subsection{Złożone struktury.}
Używając zdefiniowanych przez nas prostych struktur algebraicznych,
możemy definiować bardziej złożone,
składające się z wielu działań (ale nadal tylko jednego zbioru).

\begin{definition}
	\textbf{Pierścieniem} nazywamy trójkę $(P, +, \cdot)$, że
\end{definition}

\begin{definition}
	\textbf{Ciałem} nazywamy pierścień $(P, +, \cdot)$
\end{definition}

\subsection{Zadania.}

\begin{exercise}
	Sprawdź, czy zbiór macierzy $\mathbb{R}^{2 \times 2}$ jest grupą.
\end{exercise}

\begin{exercise}
	Podaj przykład nieprzemiennej grupy.
\end{exercise}

\begin{exercise}
	Sprawdź, czy $(\mathbb{C}, \cdot)$ jest grupą.
\end{exercise}

\begin{exercise}
	Czy $\mathbb{C}$ ze standardowym mnożeniem jest monoidem? Jeśli tak, to jaki jest element neutralny? Czy jest to grupa?
\end{exercise}

\begin{exercise}
	Niech $A$ będzie dowolnym zbiorem.
	Które z definicji struktur algebraicznych spełnia zbiór funkcji $A \to A$ z operacją składania funkcji?
\end{exercise}

\begin{exercise}
	Sprawdź, które z definicji struktur algebraicznych spełnia zbiór ścieżek na płaszczyźnie (t.j. funkcji $[0, 1] \to \mathbb{R}^2$) z operacją łączenia ich swoimi końcami (tzn. dwie ścieżki możemy połączyć w jedną, wtedy i tylko wtedy gdy pierwsza kończy się tam, gdzie druga się zaczyna).
\end{exercise}

\begin{exercise}
    Niech $(A, \cdot, e, (\cdot^{-1}))$ będzie dowolną grupą.
    Udowodnij, że dla każdego $x \in A$ zachodzi ${x^{-1}}^{-1} = x$.
\end{exercise}


\begin{exercise}
    Niech $(A, \cdot, e, (\cdot^{-1}))$ będzie dowolną grupą.
    Udowodnij, że jeśli dana jest funkcja $f: A \to A$, spełniająca
    \begin{equation*}
        \forall_{x\in A}\; f(x) \cdot x = e = x \cdot f(x),
    \end{equation*}
    to $f = (\cdot^{-1})$ (dwie funkcje są równe, jeśli dla każdego argumentu zwracają tę samą wartość).
    Pokazuje to, że mając dane działanie i element neutralny,
        istnieje conajwyżej jeden sposób zdefiniowania odwrotności.
\end{exercise}

\section{Algebra Uniwersalna.}
Wejdziemy teraz na poziom abstrakcji zwykle niedosięgany przez standardowe wykłady z algebry.
Szybko jednak zobaczymy,
że znacząco ułatwi nam to posługiwanie się strukturami,
które zdefiniowaliśmy.

Na czym polega problem?
Poznane przez nas struktury algebraiczne wyglądają wszystkie bardzo podobnie.
Mamy zbiór, pewne stałe i operacje wykonywane na tym zbiorze.
Możemy się już domyślać,
że będziemy mieli wiele bardzo podobnych twierdzeń i definicji dla każdej ze zdefiniowanych struktur.

Przykładowo,
w algebrze występuje pojęcie \textbf{homomorfizmu}---funkcji $h: A \to B$ między dwoma strukturami tego samego typu (np. między dwoma grupami, dwoma monoidami i t.d.),
które dodatkowo spełniają pewne warunki.
Zawsze jednym z nich jest (jeśli $(A, \star, \dots)$ i $(B, \bullet, \dots)$ są jakimiś strukturami)
\begin{equation}
	h(a\star b) = h(a) \bullet h(b).
\end{equation}
Ale definicje mają zwykle też więcej wymagań.
W przypadku monoidów $(A, \star, e_A), (B, \star, e_B)$,
chcemy aby zachodziło też
\begin{equation}
	h(e_A) = e_B.
\end{equation}
W przypadku grup podobna własność musi zachodzić dla odwracalności i t.d.

Najczęściej homomorfizmy definiuje się osobno dla każdej struktury.
W trakcie studiów spotkaliśmy się już z przykładem homomorfizmu---mowa o przekształceniach liniowych (przestrzenie liniowe też są strukturą algebraiczną!).
Nasuwa się jednak pytanie: czy nie dałoby się uogólnić tego, abyśmy mieli tylko jedną definicję? Chcielibyśmy mieć coś takiej postaci

\noindent \textbf{Definicja}. \textbf{Homomorfizmem} struktur $(A, \dots)$ i $(B, \dots...)$ nazywamy funkcję $h: A \to B$, spełniającą warunki \dots

Ciężko jednak powiedzieć, co należałoby wstawić w miejsce kropek, aby sformalizować tę definicję.
Podobna sytuacja zachodzi,
    jeśli zechcemy zacząć definiować \textbf{podgrupy}, \textbf{podciała}, \textbf{podmonoidy} i t.d.
Definicje tych obiektów byłyby do siebie bardzo podobne.
Jak jednak zawrzeć je wszystkie w jednej definicji?
Odpowiedzi na te pytania dostarcza nam dziedzina algebry zwana \textit{Algebrą Uniwersalną}.

\subsection{Podstawowe definicje.}

\begin{definition}
	Niech $A$ będzie zbiorem.
	Mówimy, że funkcja $f: A^n -> A$ ma \textbf{arność} $n$ (z angielskiej końcówki \textit{-ary}, a to z łacińskiej końcówki \textit{-aris}, w analogii do bin\textit{ary}, tern\textit{ary}, i t.d.).
\end{definition}

Ciekawa i przydatna rzecz zachodzi, gdy w powyższej definicji $n = 0$.
Zbiór $A^0$ definiuje się jako zbiór jednoelementowy (co jest tym jednym elementem nie ma znaczenia).
Funkcja $f:A^0 -> A$ jest więc funkcją ze zbioru jednoelementowego w zbiór $A$.
Możemy więc na nią patrzeć jak na pojedynczą stałą (bo zbiór wartości też jest jednoelementowy).

\begin{definition}
	\textbf{Algebrą} nazywamy trójkę $(\Sigma, \alpha, E)$,
	gdzie
	\begin{itemize}
		\item $\Sigma$ jest zbiorem, zwanym zbiorem \textbf{symboli},
		\item $\alpha: \Sigma \to \mathbb{N}$ jest funkcją, która każdemu symbolowi przypisuje arność,
		\item $E$ jest zbiorem równań używających zmienny i symboli z $\Sigma$ jako funkcji.
	\end{itemize}
	Parę $(\Sigma, \alpha)$ nazywamy \textbf{sygnaturą} algebry $(\Sigma, \alpha, E)$.
\end{definition}

Definicja ta nie jest w pełni formalna (czym są ,,symbole"? czym są ,,zmienne"? co znaczy, że elementów $\Sigma$ używamy jako funkcji? jak możemy czemuś, co nie jest funkcją, przypisać arność?).
Sformalizowanie jej jednak wprowadziłoby więcej zamieszania niż pożytku,
gdyż wymagałoby wprowadzenia mało istotnych, technicznych szczegółów.
Zamiast tego pokażemy następujący,
rozjaśniający przykłady.

\begin{example}
	Trójka $(\Sigma, \alpha, E)$, gdzie
	\begin{itemize}
		\item $\Sigma = \{f, g, e\}$,
		\item $\alpha: f \mapsto 2, g \mapsto 2, e \mapsto 0$,
		\item E = \{\\
		      $f(e, x) = x$ \\
		      $f(x, e) = x$ \\
		      $g(x, y) = g(y, x)$ \\
		      \}
	\end{itemize}
	jest algebrą.
\end{example}
Jak możemy interpretować, czym jest taka algebra?
Jest to nic innego jak definicja pewnej struktury algebraicznej,
gdzie mamy dwie operacje dwuargumentowe $f, g$ i jedną wyróżnioną stałą $e$,
że $g$ jest przemienne, a $e$ jest elementem neutralnym dla $f$.

Tłumacząc powyższy zapis na tradycyjne definicje, moglibyśmy napisać następującą definicję.
\begin{definition}
	\textbf{Zmyślonastrukturą} nazywamy czwórkę $(A, f, g, e)$, gdzie $A$ jest zbiorem, $f, g: A^2 \to A$ są funkcjami dwuargumentowymi, a $e \in A$, że spełnione są następujące warunki:
	\begin{itemize}
		\item $\forall_{x\in A}\; f(e, x) = x,$
		\item $\forall_{x\in A}\; f(x, e) = x,$
		\item $\forall_{x,y\in A}\; g(x, y) = g(y, x).$
	\end{itemize}
\end{definition}

Zwykle zapis będziemy jednak upraszczać, podając jednocześnie zbiór symboli $\Sigma$ i funkcję $\alpha$ za jednym razem, n.p. w postaci
\begin{equation}
	\Sigma = \{f^2, g^2, e\},
\end{equation}
t.j. podając w definicji symboli ich arność w indeksie górnym, i pomijając ją dla $0$.

Algebry odpowiadają więc definicjom struktur.
Co odpowiada jednak obiektom, spełniającym te definicje?

\begin{definition}
	\textbf{Modelem algebry} $(\Sigma, \alpha, E)$ nazywamy zbiór $A$,
	wraz z interpretacją symboli z $\Sigma$ jako funkcje o arnościach definiowanych przez $\alpha$ i spełniających równania $E$.
\end{definition}
Znowu nie jest to całkowicie formalna definicja,
na nasze potrzeby jednak w pełni wystarczy.

Możemy teraz definiować nasze struktury w taki sam sposób jak zmyślonastrukturę.
\begin{definition}
	\textbf{Monoidem} nazywamy dowolny model algebry $(\{\star^2, e\}, \{ x \star e = x, e \star x = x \})$.
\end{definition}

Ważne, aby rozróżniać teraz dwa pojęcia: algebrą jest \textit{definicja} monoidu.
Modelem algebry jest dowolny \textit{obiekt}, który spełnia tę definicję.

\subsection{Zadania.}

\begin{exercise}
	Zapisz definicję magmy w postaci algebry.
\end{exercise}
\begin{exercise}
	Zapisz definicję półgrupy w postaci algebry.
\end{exercise}
\begin{exercise}
	Zapisz definicję grupy w postaci algebry.
\end{exercise}
\begin{exercise}
	Zapisz definicję grupy abelowej w postaci algebry.
\end{exercise}

\section{Pewne definicje związane z algebrami.}

\begin{definition}
    \textbf{Podmodelem algebry} nazywamy
\end{definition}

\begin{definition}
    \textbf{Homomorfizmem} (lub czasami \textbf{morfizmem}) nazywamy
\end{definition}

Z pewnych niewyjaśnionych powodów,
    w przypadku homomorfizmów nie używa się pojęć takich jak iniekcja, bijekcja i surjekcja.
Zamiast tych pochodzących z francuskiego wyrazów mamy greckie odpowiedniki.

\begin{definition}
    Homomorfizm nazywamy
    \begin{itemize}
        \item \textbf{epimorfizmem}, jeśli jest surjekcją,
        \item \textbf{monomorfizmem}, jeśli jest iniekcją,
        \item \textbf{izomorfizmem}, jeśli jest bijekcją,
        \item \textbf{endomorfizmem}, jeśli jego dziedzina i kodziedzina są równe, t.j. jest postaci $h:A \to A$,
        \item \textbf{automorfizmem}, jeśli jest endomorfizmem i izomorfizmem.
    \end{itemize}
\end{definition}

\begin{definition}
    Wolnym modelem generowanym przez zbiór 
\end{definition}

\begin{definition}
    Prezentacją modelu
\end{definition}

\section{Twierdzenia o homomorfizmach.}

\section{Grupy.}

Pochylimy się teraz nieco bardziej szczegółowo nad grupami,
gdyż stanowią one jedną z najważniejszych i najgłębiej badanych struktur algebraicznych.



\section{Przestrzenie liniowe.}

\end{document}
